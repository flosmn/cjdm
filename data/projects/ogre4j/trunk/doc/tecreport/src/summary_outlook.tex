\section{Summary}
\label{sec:Summary}
The development of ogre4j can now be done in parallel to the 
development/release process of OGRE. The automatic code generation of the JNI 
bindings reduces the effort to maintain API breaking changes of OGRE and 
reduces the possibility of bugs in the wrapper. The development of the 
necessary XSLT style sheets resulted in a stand alone project called XBiG which 
is maintained on its project space at Sourceforge.net. The usage of XSLT has a 
lot of of advantages but for big APIs the generation of the bindings is slow 
because it is a interpreted language.

The current release ``ogre4j 1.4.3 beta2'' is available on Sourceforge.net and
should be downloaded by interested developers to test the runtime behavior of
the bindings. The bindings are yet not finished but the first samples and
demos show that it is possible to create 3D applications with it. 

\section{Outlook}
\label{sec:Outlook}
So, what are the future plans for ogre4j and XBiG? At the moment there are 
still some API parts of OGRE on the ignore list of the generation process. One 
of the next goals will be to reduce this ignore list to generated the OGRE 
bindings up to 99\%. To demonstrate the power of ogre4j we are going to migrate 
all OGRE demos to Java to have the same suite of applications which will show 
the capabilities of OGRE in Java. Parallel to the development of more samples 
and demonstration we are working at a 3D UI toolkit based on ogre4j which will 
be called ogreface. ogreface is inspired by the UI toolkit JFace and will 
provide developers a higher level API to create windowed ogre4j applications. 
Additionally there will be Eclipse plug-ins which integrate ogre4j and ogreface 
in the popular application framework.

