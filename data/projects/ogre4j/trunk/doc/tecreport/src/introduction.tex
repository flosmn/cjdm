\section{Introduction}
\label{Introduction}
ogre4j is a project that enables the use of the C++ library 
OGRE3D\footnote{Object-Oriented Graphics Rendering Engine} in Java 
applications. More precisely it is a thin layer of C and Java code on top of the 
public API of OGRE. This layer consists of the Java Native Interfaces to access 
the C/C++ code. JNI was introduced with the release of Java version 1.2 and 
gives developers the opportunity to reuse C/C++ code in their Java applications. 

The first approach was made by Ivica Aracic. He developed the first version of 
ogre4j in the year 2002. At that time OGRE was in the fledgling stages too. 
Today OGRE is one of the most popular and feature rich open source 3D render 
engines. The current version consists of about 540 classes and 18.000 lines of 
code for the main library plus os dependent and sample code.

In 2005 the netAllied GmbH decided to use OGRE as 3D render 
engine in their product portfolio which is based on the Eclipse 
RCP\footnote{Rich Client Platform} framework. This framework is mostly 
developed in Java but the main UI\footnote{User Interface} tool kit is based on 
the C/C++ APIs\footnote{Advanced Programming Interface} of the operating 
system. 

The Standard Widget Toolkit, SWT, uses Java Native Interfaces to access 
the native\footnote{Native to the operating system.} UI APIs to create a look 
and feel that isn't different to real C/C++ applications. SWT as good example 
and the performance it demonstrates assured the decision to enhance ogre4j.

