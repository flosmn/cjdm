\section{Java Native Interface}
\label{sec:JavaNativeInterface}
Java Native Interface (JNI) is a programming framework that allows Java code 
running in the Java virtual machine (VM) to call and be called by native 
applications\footnote{Programs specific to a hardware and operating system 
platform} and libraries written in other languages, such as C, C++ and assembly.

The JNI is used to write native methods to handle situations when an 
application cannot be written entirely in the Java programming language such as 
when the standard Java class library does not support the platform-specific 
features or program library. It is also used to modify an existing application, 
written in another programming language, to be accessible to Java applications. 
Many of the standard library classes depend on the JNI to provide functionality 
to the developer and the user, e.g. I/O file reading and sound capabilities. 
Including performance- and platform-sensitive API implementations in the 
standard library allows all Java applications to access this functionality in a 
safe and platform-independent manner.

The JNI framework lets a native method utilize Java objects in the same way 
that Java code uses these objects. A native method can create Java objects and 
then inspect and use these objects to perform its tasks. A native method can 
also inspect and use objects created by Java application code.

JNI is sometimes referred to as the "escape valve" for Java developers because 
it allows them to add functionality to their Java Application that the Java API 
can't provide. It can be used to interface with code written in other 
languages, like C++. It is also used for time-critical calculations or 
operations like solving complicated mathematical equations, since native code 
can potentially be faster than JVM code.

The JNI is not trivial and requires a considerable effort to learn, and some 
people recommend that only advanced programmers should use the JNI. However, 
the capability for Java to communicate with C++ and assembly removes any 
limitations on what function Java programs can perform. Programmers considering 
using the JNI should be aware that
\begin{enumerate}
	\item as mentioned before, the JNI is not an easy API to learn;
	\item only applications and signed applets can invoke the JNI;
	\item an application that relies on JNI loses the platform portability Java 
	  	offers (a workaround is to write a separate implementation of the JNI code
	 	for each platform and have Java detect the Operating System and load the correct 
	 	one at runtime)
	\item there is no garbage collection for the JNI side (JNI code must do 
		explicit deallocation);
	\item error checking is a MUST or it has the potential to crash the JNI side 
		and the JVM.
\end{enumerate}